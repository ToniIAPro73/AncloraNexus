\documentclass[12pt,a4paper]{article}

% Paquetes esenciales
\usepackage[utf8]{inputenc}
\usepackage[T1]{fontenc}
\usepackage[spanish]{babel}
\usepackage{amsmath}
\usepackage{amsfonts}
\usepackage{amssymb}
\usepackage{graphicx}
\usepackage{cite}
\usepackage{url}
\usepackage{hyperref}
\usepackage{geometry}
\usepackage{setspace}

% Configuración de página
\geometry{margin=2.5cm}
\onehalfspacing

% Metadatos del documento
\title{Análisis Comparativo de Algoritmos de ML}
\author{Dr. Investigador}
\date{\today}

% Configuración de hyperref
\hypersetup{
    colorlinks=true,
    linkcolor=blue,
    filecolor=magenta,      
    urlcolor=cyan,
    citecolor=red
}

\begin{document}

\maketitle
\tableofcontents
\newpage


Este estudio presenta un análisis exhaustivo de diferentes algoritmos de aprendizaje automático aplicados a problemas de clasificación.


INTRODUCCIÓN

El machine learning ha revolucionado múltiples campos de la ciencia y la tecnología.


\section{Objetivos del Estudio}

\begin{enumerate}
\item Comparar la eficiencia de algoritmos supervisados
\item Evaluar la precisión en diferentes datasets
\item Analizar la complejidad computacional
\end{enumerate}


METODOLOGÍA

\section{Se utilizaron los siguientes algoritmos}

\begin{itemize}
\item Support Vector Machines (SVM)
\item Random Forest
\item Neural Networks
\item Naive Bayes
\end{itemize}


Métricas de Evaluación

\begin{quote}
precision = TP / (TP + FP)
recall = TP / (TP + FN)
f1\_score = 2 \textit{ (precision } recall) / (precision + recall)
\end{quote}


\section{RESULTADOS}

Los experimentos mostraron que \textbf{Random Forest} obtuvo el mejor rendimiento promedio.


Análisis Estadístico

La \textit{significancia estadística} fue evaluada usando pruebas t-student.


\section{CONCLUSIONES}

Los resultados sugieren que la elección del algoritmo depende del contexto específico.


\section{REFERENCIAS}

\cite{Mitchell1997} Machine Learning

\cite{Bishop2006} Pattern Recognition and Machine Learning


\end{document}